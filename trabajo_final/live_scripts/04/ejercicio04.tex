% This LaTeX was auto-generated from MATLAB code.
% To make changes, update the MATLAB code and export to LaTeX again.

\documentclass{article}

\usepackage[utf8]{inputenc}
\usepackage[T1]{fontenc}
\usepackage{lmodern}
\usepackage{graphicx}
\usepackage{color}
\usepackage{listings}
\usepackage{hyperref}
\usepackage{amsmath}
\usepackage{amsfonts}
\usepackage{epstopdf}
\usepackage{matlab}

\sloppy
\epstopdfsetup{outdir=./}
\graphicspath{ {./ejercicio04_images/} }

\begin{document}

\matlabtitle{Ejercicio Nº4}


\begin{matlabcode}
clear;clc;
\end{matlabcode}

\begin{par}
\begin{flushleft}
Se definen simbólicas las variables
\end{flushleft}
\end{par}

\begin{matlabcode}
syms t il1(t) il2(t) il3(t) L1 L2 L3 R1 R2 E;
\end{matlabcode}

\begin{par}
\begin{flushleft}
Se plantean las ecuaciones y se obtienen las matrices de la forma generalizada
\end{flushleft}
\end{par}

\begin{matlabcode}
M=[L1 L1+L3;L2 -L3]
\end{matlabcode}
\begin{matlabsymbolicoutput}
M = 
    $\displaystyle \left(\begin{array}{cc}
L_1  & L_1 +L_3 \\
L_2  & -L_3 
\end{array}\right)$
\end{matlabsymbolicoutput}
\begin{matlabcode}
N=[R1 R1;R2 0]
\end{matlabcode}
\begin{matlabsymbolicoutput}
N = 
    $\displaystyle \left(\begin{array}{cc}
R_1  & R_1 \\
R_2  & 0
\end{array}\right)$
\end{matlabsymbolicoutput}
\begin{matlabcode}
u=[E;0]
\end{matlabcode}
\begin{matlabsymbolicoutput}
u = 
    $\displaystyle \left(\begin{array}{c}
\textrm{E}\\
0
\end{array}\right)$
\end{matlabsymbolicoutput}

\begin{par}
\begin{flushleft}
Se expresan las matrices de la forma normalizada
\end{flushleft}
\end{par}

\vspace{1em}

\begin{matlabcode}
A=-1.*(M\N)
\end{matlabcode}
\begin{matlabsymbolicoutput}
A = 
    $\displaystyle \left(\begin{array}{cc}
-\frac{L_1  R_2 +L_3  R_1 +L_3  R_2 }{L_1  L_2 +L_1  L_3 +L_2  L_3 } & -\frac{L_3  R_1 }{L_1  L_2 +L_1  L_3 +L_2  L_3 }\\
\frac{L_1  R_2 -L_2  R_1 }{L_1  L_2 +L_1  L_3 +L_2  L_3 } & -\frac{L_2  R_1 }{L_1  L_2 +L_1  L_3 +L_2  L_3 }
\end{array}\right)$
\end{matlabsymbolicoutput}
\begin{matlabcode}

B=M\u
\end{matlabcode}
\begin{matlabsymbolicoutput}
B = 
    $\displaystyle \left(\begin{array}{c}
\frac{\textrm{E} L_3 }{L_1  L_2 +L_1  L_3 +L_2  L_3 }\\
\frac{\textrm{E} L_2 }{L_1  L_2 +L_1  L_3 +L_2  L_3 }
\end{array}\right)$
\end{matlabsymbolicoutput}

\begin{par}
\begin{flushleft}
Se definen las variables de estado
\end{flushleft}
\end{par}

\begin{matlabcode}
x=[il2;il3]
\end{matlabcode}
\begin{matlabsymbolicoutput}
x(t) = 
    $\displaystyle \left(\begin{array}{c}
{\textrm{il}}_2 \left(t\right)\\
{\textrm{il}}_3 \left(t\right)
\end{array}\right)$
\end{matlabsymbolicoutput}

\begin{par}
\begin{flushleft}
Expresando el sistema en forma diferencial
\end{flushleft}
\end{par}

\begin{matlabcode}
odes = diff(x) ==  A*x + B
\end{matlabcode}
\begin{matlabsymbolicoutput}
odes(t) = 
    $\displaystyle \begin{array}{l}
\left(\begin{array}{c}
\frac{\partial }{\partial t}\;{\textrm{il}}_2 \left(t\right)=\frac{\textrm{E} L_3 }{\sigma_1 }-\frac{{\textrm{il}}_2 \left(t\right) {\left(L_1  R_2 +L_3  R_1 +L_3  R_2 \right)}}{\sigma_1 }-\frac{L_3  R_1  {\textrm{il}}_3 \left(t\right)}{\sigma_1 }\\
\frac{\partial }{\partial t}\;{\textrm{il}}_3 \left(t\right)=\frac{{\textrm{il}}_2 \left(t\right) {\left(L_1  R_2 -L_2  R_1 \right)}}{\sigma_1 }+\frac{\textrm{E} L_2 }{\sigma_1 }-\frac{L_2  R_1  {\textrm{il}}_3 \left(t\right)}{\sigma_1 }
\end{array}\right)\\
\mathrm{}\\
\textrm{where}\\
\mathrm{}\\
\;\;\sigma_1 =L_1  L_2 +L_1  L_3 +L_2  L_3 
\end{array}$
\end{matlabsymbolicoutput}

\begin{par}
\begin{flushleft}
Resolviendo el sistema con el comando dsolve
\end{flushleft}
\end{par}

\begin{matlabcode}
[i2Sol(t), i3Sol(t)] = dsolve(odes);
\end{matlabcode}

\begin{par}
\begin{flushleft}
Reemplazando los valores de E,R1, R2, L y C
\end{flushleft}
\end{par}

\begin{matlabcode}
clear L1 L2 L3 R1 R2 E;
syms C1 C2;
R1=1;R2=1;L1=1;L2=1;L3=2;
E=2*exp(-2*t);
A=subs(A);
B=subs(B);
odes = diff(x) == A*x + B;
\end{matlabcode}

\matlabheading{Estableciendo condiciones iniciales y tiempo de simulación}

\begin{matlabcode}
i0l2=0;
i0l3=0;
ti=0;
tf=4*pi;
Xant=[i0l2;i0l3];
constantes=x(0)==Xant;
[i2Sol(t), i3Sol(t)] = dsolve(odes,constantes);
\end{matlabcode}

\matlabheading{Voltaje en la resistencia R2}

\begin{matlabcode}
VR2=simplify(i2Sol*R2)
\end{matlabcode}
\begin{matlabsymbolicoutput}
VR2(t) = 
    $\displaystyle -\frac{e^{-2 t}  {\left(e^{\frac{9 t}{5}} -9 e^t +8\right)}}{9}$
\end{matlabsymbolicoutput}

\end{document}
