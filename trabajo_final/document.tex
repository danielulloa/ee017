\documentclass[10pt,a4paper]{article} % tamaño de letra y tipo de papel
\usepackage[utf8]{inputenc}
\usepackage[spanish]{babel} % paquete para que reconozca ñ y tildes
\usepackage{amsmath}
\usepackage{upgreek} 
\usepackage{amsfonts}
\usepackage{amssymb}
\usepackage{graphicx} % paquete para incluir imagenes
\graphicspath{ {imagenes/}}
\usepackage[margin=1in,bottom=1in]{geometry}
\usepackage{hyperref} % paquete para tener marcadores en el pdf
\usepackage{tikz}
\usetikzlibrary{babel}
\usepackage[siunitx, RPvoltages]{circuitikz}
\usetikzlibrary{bending,arrows.meta,positioning,calc,positioning}
\usepackage{pgfplots}\pgfplotsset{compat=1.13}
\usepackage{float}
\usepackage[T1]{fontenc}
\usepackage[numbered,framed]{matlab-prettifier}
\pgfplotsset{width=12cm,legend style={at={(0.11,0.75)},anchor=south},select coords between index/.style 2 args={
		x filter/.code={
			\ifnum\coordindex<#1\def\pgfmathresult{}\fi
			\ifnum\coordindex>#2\def\pgfmathresult{}\fi
		}
}}
\author{Ulloa Daniel & Rodriguez Victoria}
\begin{document}
	\begin{titlepage}
		\hbox{
			\hspace*{0.15\textwidth} % Espacio desde el margen izquierdo
			\rule{1pt}{\textheight} % Linea decorativa
			\hspace*{0.05\textwidth} % Espacio entre la linea y el texto
			\parbox[b]{0.75\textwidth}{ % Caja que restringe el espacio que puede ocupar el texto
				{\noindent\Huge\bfseries Trabajo Final } % Titulo
				\\ 
				[2\baselineskip] 
				{\large \textbf{Tema:} Dinámica de Circuitos} % Tema
				\\[4\baselineskip]
				{\large \textbf{Cátedra:} Teoría de Circuitos \textsc{II}} % Catedra
				\\[1\baselineskip]
				{\large \textbf{Año:} 2020} % Año
				\\[1\baselineskip]
				{\large \textit{\textbf{Docentes:} % Docentes
						\textnormal{Ing. Pires}, Eduardo. 
						\textnormal{Ing. Costa}, Nicolás}
				}
				\\[1\baselineskip]
				{\large \textit{\textbf{Alumnos:} % Alumnos
						\textnormal{Rodriguez}, Ana Victoria. 
						\textnormal{Ulloa}, Daniel Alejandro}
				}
				\\[6\baselineskip]
				{\large \textbf{Fecha:} 13/02/2020}
				\par %Para que el logo aparezca al pie
				\vspace{0.35\textheight} % Ubicacion de la caja desde el margen superior
				\center{\includegraphics[width=250px]{logo2.png}}
				\\[1\baselineskip]
		}}
	\end{titlepage}
	\tableofcontents
	\newpage
	\section{Introducción}
	Este trabajo se enfoca en estudiar la dinámica de circuitos presentados en la Unidad 3 del libro $\textsc{Classical Circuit Theory}$, para esto es necesario encontrar la solución de Sistemas de Ecuaciones Diferenciales Algebraicas en forma analítica y aplicando los métodos numéricos de Euler.\\
	
	Analizar los circuitos a partir de sus ecuaciones de estado permite obtener la respuesta transitoria y estacionaria, mientras que trabajando en el plano de Laplace sólo obtenemos la respuesta de estado estacionario y únicamente es válido si las condiciones iniciales son nulas.\\
	
	A partir de las trayectorias de estado en distintos planos (X-Y, Y-Z, X-Z) es posible representar la relación existente entre las variables de estado del circuito, por ejemplo representar corriente versus tensión, a este tipo de diagramas se los conoce como $\textsc{Phase Portrait}$. Estas trayectorias dependen de las condiciones iniciales.	
	
	\subsection{Obtención de las ecuaciones de estado}
	
	Representando las ecuaciones de nodos modificados de la siguiente forma:
	
	\begin{equation}
		M\frac{d\vec{x}(t)}{dt}+N\vec{x}(t)=E\textbf{u}(t)\label{eq:general}
	\end{equation}
	
	podemos observar que el vector $\vec{x}$(t) está compuesto por las variables de estado, $M$ es la matriz que expresa las relaciones constitutivas de los componentes dinámicos, $N$ es la matriz de admitancias, $E$ una matriz de fuentes y $\textbf{u}(t)$ una función vectorial.\\
	
	Despejando $\frac{d\vec{x(t)}}{dt}$ de \ref{eq:general}:
	\begin{align}
		M^{-1}M\frac{d\vec{x}(t)}{dt}&=M^{-1}\left(E\textbf{u}(t)-N\vec{x}(t)\right)\nonumber\\
		I\frac{d\vec{x}(t)}{dt}&=M\backslash E\textbf{u}(t)- M\backslash N\vec{x}(t)\label{eq:mizquierda}
	\end{align}
	
	De \ref{eq:mizquierda} se obtiene la expresión:
	
	\begin{equation}
		\frac{d\vec{x}(t)}{dt}=A\vec{x}(t)+B\textbf{u}(t) \label{eq:normalizada}
	\end{equation}
	
	Resolviendo \ref{eq:normalizada} se obtiene $\vec{x}(t)$ que satisface \ref{eq:general}\\
	
	Para expresar las salidas del circuito es necesario que estén en función de las variables de estado y se consideren las fuentes de excitación:
	
		\begin{equation}
	\vec{y}(t)=C\vec{x}(t)+Du(t)\label{eq:salida}
	\end{equation}
	
	Ahora en \ref{eq:salida} la función vectorial $\textbf{u}(t)$ queda expresada como una función escalar $u(t)$
	
	\subsection{Solución analítica utilizando Matlab}
	
	
	utilizando matlab y el comando dsolve...
	
	va el script
	
	\subsection{Solución por método numérico}
	Partiendo de \ref{eq:mizquierda} la derivada en un tiempo $t_n$ se aproxima por la pendiente de una linea recta pasando por la incógnita $\vec{x}_n$ y su último valor conocido $\vec{x}_{n-1}$:
	\begin{equation}
		\frac{d\vec{x}}{dt}\Bigr\rvert_{t_n}\approx\frac{\vec{x}_n-\vec{x}_{n-1}}{h}
	\end{equation}
	
	Se obtiene el método \textsc{Backward Euler} en forma vectorial:
	\begin{align}
		\vec{x}_n&=\left[\frac{1}{h}M+N\right]\backslash \textbf{u}(t_n)+\left[\frac{1}{h}M+N\right]\backslash \left(\frac{1}{h}M \vec{x}_{n-1}\right)\label{euler}
	\end{align}
	
	aca va el script...
	
	
	\subsection{Solución de un circuito}
	La respuesta temporal de la tensión de salida $V_R$ del siguiente circuito RLC se puede representar utilizando las soluciones del sistema \ref{eq:normalizada}, para esto debemos expresar las matrices C y D de la ecuacion \ref{eq:salida}
		
		\begin{center}
			\begin{circuitikz}[american voltages]\label{fig1}
				\ctikzset{label/align = smart}
				\draw (0,0) node[ground]{} 
				(0,0) to [V,label=E(t)](0,2)
				(0,2) to [L, label=$L$]++(2,0) to [short,-*]++(0,0) coordinate (nodo1) to [C,l=C]++(0,-2) node[ground]{}
				(nodo1) to [short,-]++(1.5,0) to [R,l=$R$]++(0,-2) node[ground]{}
				;
			\end{circuitikz}
			\\ Figura \ref{fig1}
		\end{center}
	Planteando las ecuaciones y ordenandolas con la forma de \ref{eq:general}:
	\begin{align}
		C\frac{dV_C}{dt}+\frac{V_C}{R}-il&=0\\
		-L\frac{di_L}{dt}+\frac{V_C}{R}&=0		
	\end{align}
	
	
	Expresando en forma matricial:
	\begin{equation}
		\begin{pmatrix}
		0&L\\
		C&0
		\end{pmatrix}\frac{d}{dt}\begin{bmatrix}
		V_C\\
		i_L
		\end{bmatrix}+\begin{pmatrix}
		-1 & 0\\
		\frac{1}{R}&1
		\end{pmatrix}\begin{bmatrix}
		V_C\\
		i_L
		\end{bmatrix}=\begin{pmatrix}
		0\\
		0
		\end{pmatrix}\textbf{u}(t)
	\end{equation}
	
	El vector salida es:
	\begin{equation}
	V_R=\begin{pmatrix}
	1&0
	\end{pmatrix}\begin{bmatrix}
	V_c\\
	i_L
	\end{bmatrix}+\begin{pmatrix}
	0\\
	0
	\end{pmatrix}\textbf{u}(t)
	\end{equation}
	
Las soluciones son tal .... 



	
	
	
	
	
	
	 
	\section{Guía de Problemas}
	\textbf{1} Escribir las ecuaciones de estado de un circuito formado por un inductor $L$ en paralelo con un capacitor $C$. Obtener la solución en términos de la corriente inicial del inductor $i_L(0)$ y del voltaje inicial del capacitor $v_C(0)$. Mostrar que la trayectoria es una elipse en el espacio de estados.\\
	
	\textbf{2} Mostrar que los valores propios del circuito de la Figura \ref{fig2} son $-1\pm j$. Encontrar la solución completa para condiciones iniciales arbitrarias y una excitación arbitraria $E(t)$. Sea $C=1F$, $L=1H$, $R_1=R_2=1\Omega$. Graficar la trayectoria de la solución homogénea para dos condiciones iniciales en el espacio de estados.\\
	 \begin{center}
		\begin{circuitikz}[american voltages]\label{fig2}
			\ctikzset{label/align = smart}
			\draw (0,0) node[ground]{} 
			(0,0) to [V,label=E(t)](0,2)
			(0,2) to [R, label=$R_1$]++(2,0) to [short,-*]++(0,0) coordinate (nodo1) to [C,l=C]++(0,-2) node[ground]{}
			(nodo1) to [L,label=L]++(2,0) to [R,l=$R_2$]++(0,-2) node[ground]{}
			;
		\end{circuitikz}
	\\ Figura \ref{fig2}
	\end{center}
	
	\textbf{3} Para el circuito de la Figura \ref{fig3}, $C_1=C_2=C_3=1F$, $R_1=R_2=1\Omega$. Mostrar que los valores propios son $-1$ y $-\frac{1}{3}$. Asumir que la excitacion $E(t)=10\cos(\omega t)$. Encontrar la respuesta de estado estacionario.
	\begin{center}
		\begin{circuitikz}[american voltages]\label{fig3}
			\ctikzset{label/align = smart}
			\draw (0,0) node[ground]{} 
			(0,0) to [V,label=E(t)](0,2)
			(0,2) to [R, label=$R_1$]++(2,0) to [short,-*]++(0,0) coordinate (nodo1) to [C,l=$C_1$]++(0,-2) node[ground]{}
			(nodo1) to [C,label=$C_3$]++(2,0) to [short,-*]++(0,0) coordinate (nodo2) to [C,l=$C_2$]++(0,-2) node[ground]{}
			(nodo2) to [short]++(2,0) to [R,label=$R_2$]++(0,-2) node[ground]{}
			;
		\end{circuitikz}
		\\ Figura \ref{fig3}
	\end{center}

	\textbf{4} En el circuito de la figura, sea $v_{out}(t)$ el voltaje a traves de la resistencia $R_2$ y $E(t)=2e^-2t$ para $t\geq 0$ y $E(t)=0$ caso contrario. Mostrar que:\\
	
	$v_{out}(t)= \left\{ \begin{array}{lcc}
		e^{-t}-\dfrac{8}{9}e^{-t}-\dfrac{1}{9}e^{-\frac{1}{5}t} &   si  & t \geq 0 \\
		\\ 0 &   & otros\ casos.
	\end{array}
	\right.$
	
	\textbf{5} La fuente $E(t)$ del circuito de la figura se define como $E(t)=1V\ \forall t\leq 0$ y caso contrario $E(t)=0$. Mostrar que el valor a través de la resistencia $R_2$ para $t\geq 0$ es
	\begin{equation}
		v_2(t)=\frac{1}{2}e^{-t}+\frac{\sqrt{3}}{3}e^{-\dfrac{t}{2}}\sin \frac{\sqrt{3}}{3}t
	\end{equation}
	Los valores de los elementos son $R_1=R_2=1\Omega$,$C_1=C_2=1F$ y $L=2H$. Graficar la salida $v_2(t)$ para el intervalo de tiempo $0\leq t \leq 10s$.\\
	
	\textbf{6} Aplicar el metodo $Backward\ Euler$ para resolver las ecuaciones de estado del problema anterior siendo $E(t)=\sin t + r(t)$ dónde $r(t)$ es un ruido aleatorio cuya amplitud se encuentra uniformemente distribuida en el rango $[-0.1,0.1]$. Graficar la salida.\\
	
	\textbf{7} En el circuito de la figura, suponer que el voltaje inicial del capacitor $C_1$ es $1V$, y que todas las condiciones iniciales restante son nulas. Mostrar que el voltaje a traves de $g_4$ para todo $t\geq 0$ está dado por la siguiente ecuación:
	\begin{equation}
		v_{4_n}(t)=0.225e^{\alpha t}\cos \beta t-0.0087e^{\alpha t}\sin \beta t-0.1434e^{\lambda_3 t}-0.0791e^{\lambda_4 t}
	\end{equation}
	
	Dónde $\alpha=-0.5563$, $\beta=0.9145$, $\lambda_3=-1.1255$ y $\lambda_4=-0.6786$. Los valores de los elementos son $g_1=1S$, $g_2=2S$, $g_3=3S$, $g_4=4S$, $C_1=C_2=1F$, $L_1=L_2=1H$. Graficar la proyección de la trayectoria de estado en distintos planos $2D$ para estudiar la dinámica del circuito.\\
	
	\textbf{8} Mostar que en el circuito de la figura el voltaje a través de $R_2$ es, con una precision de 4 dígitos:
	\begin{equation}
		v_2(t)=\int_{0}^{t}\left[0.4813e^{\lambda_1(t-\tau)}-0.0440e^{\lambda_2(t-\tau)}\right]E(\tau)d\tau
	\end{equation}
	
	Dónde $\lambda_1=-0.9645$ y $\lambda_2=-0.0882$. Asumir que todas las condiciones iniciales son cero. Sea la entrada un pulso $E(t)=\sin^2(\frac{\pi t}{5})$ para el intervalo de tiempo $0\leq t\leq 5$, $E(t)=0$ caso contrario. Encontrar el valor de $v_2(t)$ para el intervalo $0\leq t\leq 10$. Usar convolución numérica y comparar la solución con la obtenida por $Backwar\ Euler$. Los valores de los elementos son $C_1=1F$, $C_2=2F$, $C_3=3F$, $C_4=4F$, $C_5=5F$, $C_6=6F$, y $R_1=R_2=1\Omega$\\
	
	\textbf{9} Mostrar que la respuesta al impulso de una escalera de 5 secciones de la figura, que modela una longitud de interconexión en un circuito integrado, teniendo en cuenta que la salida es el último nodo, que $R=1\Omega$ y $C=0.1F$ y la expresión es:
	\begin{equation}
		v_{out}(t)=0.554e^{\lambda_1 t}-1.788e^{\lambda_2 t}+2.720e^{\lambda_3 t}-2.500e^{\lambda_4 t}+1.014e^{\lambda_5 t}
	\end{equation}
	
	Dónde los valores de $\lambda_n$ son los siguientes:
	\begin{center}
		\begin{tabular}{ccc}
			$\lambda_1=-36.8250$ & $\lambda_2=-28.3083$ & $\lambda_3=-17.1537$ \\ 
			$\lambda_4=-6.9028$ & $\lambda_5=-0.8101$ &  \\
		\end{tabular} 
	\end{center}

	\textbf{10} Considerar un circuito $LC$ de cuarto orden que consiste en un inductor $L_1$ en serie con un capacitor $C_1$ y con una combinación paralelo de un inductor $L_2$ y un capacitor $C_2$. Sea $L_1=1H$, $C_1=\frac{1}{25}F$, $L_2=18H$ y $C_2=\frac{1}{72}F$. Sean las variables de estado $i_{L1}$, $i_{L2}$,$v_{C1}$, $v_{C1}$. Mostrar que las respuestas a una condición inicial $v_{C1}=1V$ son:
	\begin{center}
		\begin{tabular}{cc}
			$i_{L1}=\dfrac{-16}{165}\sin(10t)-\dfrac{1}{33}\sin(t)$& $v_{C1}=\dfrac{25}{33}\cos(t)+\dfrac{8}{33}\cos(10t)$\\
			&\\ 
			$i_{L2}=\dfrac{-4}{99}\sin(t)+\dfrac{2}{495}\sin(10t)$ & $v_{C2}=\dfrac{8}{11}\cos(10t)-\dfrac{8}{11}\cos(t)$\\
		\end{tabular} 
	\end{center}
	
\end{document} 