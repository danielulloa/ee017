% This LaTeX was auto-generated from MATLAB code.
% To make changes, update the MATLAB code and export to LaTeX again.

\documentclass{article}

\usepackage[utf8]{inputenc}
\usepackage[T1]{fontenc}
\usepackage{lmodern}
\usepackage{graphicx}
\usepackage{color}
\usepackage{listings}
\usepackage{hyperref}
\usepackage{amsmath}
\usepackage{amsfonts}
\usepackage{epstopdf}
\usepackage{matlab}

\sloppy
\epstopdfsetup{outdir=./}
\graphicspath{ {./ejercicio03_images/} }

\begin{document}

\matlabtitle{Ejercicio N°3}


\begin{matlabcode}
clear;clc;
\end{matlabcode}

\begin{par}
\begin{flushleft}
Se definen simbólicas las variables
\end{flushleft}
\end{par}

\begin{matlabcode}
syms t vc1(t) vc2(t) C1 C2 C3 R1 R2 w;
\end{matlabcode}

\begin{par}
\begin{flushleft}
Se plantean las ecuaciones y se obtienen las matrices de la forma generalizada
\end{flushleft}
\end{par}

\begin{matlabcode}
M=[C3 -C2;C1+C3 C1]
\end{matlabcode}
\begin{matlabsymbolicoutput}
M = 
    $\displaystyle \left(\begin{array}{cc}
C_3  & -C_2 \\
C_1 +C_3  & C_1 
\end{array}\right)$
\end{matlabsymbolicoutput}
\begin{matlabcode}
N=[0 -1/R2; 1/R1 1/R1]
\end{matlabcode}
\begin{matlabsymbolicoutput}
N = 
    $\displaystyle \left(\begin{array}{cc}
0 & -\frac{1}{R_2 }\\
\frac{1}{R_1 } & \frac{1}{R_1 }
\end{array}\right)$
\end{matlabsymbolicoutput}
\begin{matlabcode}
u=[0;10*cos(w*t)/R1];
\end{matlabcode}

\begin{par}
\begin{flushleft}
Se expresan las matrices de la forma normalizada
\end{flushleft}
\end{par}

\begin{matlabcode}
A=-1.*(M\N)
\end{matlabcode}
\begin{matlabsymbolicoutput}
A = 
    $\displaystyle \begin{array}{l}
\left(\begin{array}{cc}
-\frac{C_2 }{\sigma_2 } & \frac{C_1  R_1 -C_2  R_2 }{\sigma_1 }\\
-\frac{C_3 }{\sigma_2 } & -\frac{C_1  R_1 +C_3  R_1 +C_3  R_2 }{\sigma_1 }
\end{array}\right)\\
\mathrm{}\\
\textrm{where}\\
\mathrm{}\\
\;\;\sigma_1 =R_1  R_2  {\left(C_1  C_2 +C_1  C_3 +C_2  C_3 \right)}\\
\mathrm{}\\
\;\;\sigma_2 =R_1  {\left(C_1  C_2 +C_1  C_3 +C_2  C_3 \right)}
\end{array}$
\end{matlabsymbolicoutput}
\begin{matlabcode}

B=M\u
\end{matlabcode}
\begin{matlabsymbolicoutput}
B = 
    $\displaystyle \left(\begin{array}{c}
\frac{10 C_2  \cos \left(t w\right)}{R_1  {\left(C_1  C_2 +C_1  C_3 +C_2  C_3 \right)}}\\
\frac{10 C_3  \cos \left(t w\right)}{R_1  {\left(C_1  C_2 +C_1  C_3 +C_2  C_3 \right)}}
\end{array}\right)$
\end{matlabsymbolicoutput}

\begin{par}
\begin{flushleft}
Se definen las variables de estado
\end{flushleft}
\end{par}

\begin{matlabcode}
x=[vc1;vc2]
\end{matlabcode}
\begin{matlabsymbolicoutput}
x(t) = 
    $\displaystyle \left(\begin{array}{c}
{\textrm{vc}}_1 \left(t\right)\\
{\textrm{vc}}_2 \left(t\right)
\end{array}\right)$
\end{matlabsymbolicoutput}

\begin{par}
\begin{flushleft}
Expresando el sistema en forma diferencial
\end{flushleft}
\end{par}

\begin{matlabcode}
odes = diff(x) ==  A*x + B
\end{matlabcode}
\begin{matlabsymbolicoutput}
odes(t) = 
    $\displaystyle \begin{array}{l}
\left(\begin{array}{c}
\frac{\partial }{\partial t}\;{\textrm{vc}}_1 \left(t\right)=\frac{10 C_2  \cos \left(t w\right)}{R_1  \sigma_1 }-\frac{C_2  {\textrm{vc}}_1 \left(t\right)}{R_1  \sigma_1 }+\frac{{\textrm{vc}}_2 \left(t\right) {\left(C_1  R_1 -C_2  R_2 \right)}}{R_1  R_2  \sigma_1 }\\
\frac{\partial }{\partial t}\;{\textrm{vc}}_2 \left(t\right)=\frac{10 C_3  \cos \left(t w\right)}{R_1  \sigma_1 }-\frac{C_3  {\textrm{vc}}_1 \left(t\right)}{R_1  \sigma_1 }-\frac{{\textrm{vc}}_2 \left(t\right) {\left(C_1  R_1 +C_3  R_1 +C_3  R_2 \right)}}{R_1  R_2  \sigma_1 }
\end{array}\right)\\
\mathrm{}\\
\textrm{where}\\
\mathrm{}\\
\;\;\sigma_1 =C_1  C_2 +C_1  C_3 +C_2  C_3 
\end{array}$
\end{matlabsymbolicoutput}

\begin{par}
\begin{flushleft}
Reemplazando los valores de R1, R2 y los capacitores
\end{flushleft}
\end{par}

\begin{matlabcode}
clear C1 C2 C3 R1 R2 w;
syms C11 C12;
R1=1;R2=1;C1=1;C2=1;C3=1;w=1;
A=subs(A);
B=subs(B);
\end{matlabcode}

\matlabheading{Autovalores del circuito}

\begin{matlabcode}
autovalores= eig(A)
\end{matlabcode}
\begin{matlabsymbolicoutput}
autovalores = 
    $\displaystyle \left(\begin{array}{c}
-1\\
-\frac{1}{3}
\end{array}\right)$
\end{matlabsymbolicoutput}

\begin{par}
\begin{flushleft}
Las ecuaciones diferenciales son
\end{flushleft}
\end{par}

\begin{matlabcode}
odes = diff(x) == A*x + B
\end{matlabcode}
\begin{matlabsymbolicoutput}
odes(t) = 
    $\displaystyle \left(\begin{array}{c}
\frac{\partial }{\partial t}\;{\textrm{vc}}_1 \left(t\right)=\frac{10 \cos \left(t\right)}{3}-\frac{{\textrm{vc}}_1 \left(t\right)}{3}\\
\frac{\partial }{\partial t}\;{\textrm{vc}}_2 \left(t\right)=\frac{10 \cos \left(t\right)}{3}-\frac{{\textrm{vc}}_1 \left(t\right)}{3}-{\textrm{vc}}_2 \left(t\right)
\end{array}\right)$
\end{matlabsymbolicoutput}

\matlabheading{Estableciendo condiciones iniciales y tiempo de simulación}

\begin{par}
\begin{flushleft}
Para el primer par de condiciones iniciales
\end{flushleft}
\end{par}

\begin{matlabcode}
vc01=0;
vc02=0;
ti=0;
tf=6*pi;
Xant=[vc01;vc02];
constantes=x(0)==Xant;
[vc1Sol(t), vc2Sol(t)] = dsolve(odes,constantes)
\end{matlabcode}
\begin{matlabsymbolicoutput}
vc1Sol(t) = 
    $\displaystyle \sqrt{10} \cos \left(t-\textrm{atan}\left(3\right)\right)-\frac{1}{{{\left(e^t \right)}}^{1/3} }$
vc2Sol(t) = 
    $\displaystyle \frac{1}{2 {{\left(e^t \right)}}^{1/3} }-\frac{5 e^{-t} }{2}+\sqrt{5} \cos \left(t-\textrm{atan}\left(\frac{1}{2}\right)\right)$
\end{matlabsymbolicoutput}

\begin{par}
\begin{flushleft}
Las exponenciales se extinguen pasado cierto tiempo y la respuesta de estado estacionario es:
\end{flushleft}
\end{par}

\begin{par}
$$VC_1(t)=\sqrt{10}\cos(t-\arctan{3})$$
\end{par}

\begin{par}
$$VC_2(t)=\sqrt{5}\cos(t-\arctan{\frac{1}{2}})$$
\end{par}

\end{document}
