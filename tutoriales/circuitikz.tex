\documentclass[12pt,a4paper]{article}
\usepackage[utf8]{inputenc}
\usepackage[spanish]{babel}
\usepackage{amsmath}
\usepackage{amsfonts}
\usepackage{amssymb}
\usepackage{graphicx}
\usepackage{hyperref}
\usepackage[left=2.00cm,right=1.00cm,top=2.00cm,bottom=2.00cm]{geometry}
\usepackage{tikz}
\usetikzlibrary{babel}
\usepackage[siunitx, RPvoltages]{circuitikz}
\usetikzlibrary{bending,arrows.meta,positioning,calc,positioning}
\author{Daniel Ulloa}
\title{Informe}
\begin{document}

\section{Inicios en CircuiTikZ}

Para comenzar es necesario cargar algunos paquetes adicionales, ya que CircuiTikZ da problemas si se está utilizando con babel.
\begin{align*}
	&\backslash usepackage\{tikz\}\\
	&\backslash usetikzlibrary\{babel\}\\
	&\backslash usepackage[siunitx, RPvoltages]\{circuitikz\}	
\end{align*}

Fundamental utilizar el manual actualizado de CircuiTikz:

\url{http://mirrors.ctan.org/graphics/pgf/contrib/circuitikz/doc/circuitikzmanual.pdf}\\

\subsection{Primer circuito}
Vamos a dibujar una fuente y dos resistencias.

\begin{center}
	\begin{circuitikz}[american voltages]
		\draw (0,0) node[ground]{}; 	% Se dibuja en el origen el nodo de referencia
		\draw (0,0) to [V](0,2);		% Una fuente entre GND y un punto a 2 unidades en y, coordenada absoluta
		\draw (0,2) 					% desde el punto (x,y)=(0,2) 
			to [R]++(2,0) 				% se dibuja una resistencia horizontal a 2 unidades relativas 		
				to [R]++(0,-2) 			% se dibuja una resistencia vertical a 2 unidades relativas del elemento anterior
					node[ground]{};		% conexion a gnd
				%	to [short,-*](0,0); % una forma alternativa para conectar al origen 
	\end{circuitikz}
\end{center}

\subsection{Agregando nodos}
Para agregar nodos es útil la instrucción \textsc{coordinate} para referenciar un punto a una etiqueta, luego nos permite continuar dibujando desde ese nodo.
\begin{center}
	\begin{circuitikz}[american voltages]
		\draw (0,0) node[ground]{}; 
		\draw (0,0) to [V](0,2);
		\draw (0,2) 
			to [R]++(2,0) 
				to [short,*-]++(0,0) coordinate (nodo1)
					to [R]++(0,-2)
						node[ground]{};
		\draw (nodo1) to [short,-o]++(1,0); % conectamos un cable y una salida
	\end{circuitikz}
\end{center}

Agreguemos más nodos y diferentes componentes:

\begin{center}
	\begin{circuitikz}[american voltages]
		\draw (0,0) node[ground]{}; 
		\draw (0,0) to [sV](0,2);
		\draw (0,2) to [R]++(2,0) to [short,*-]++(0,0) coordinate (nodo1) to [C]++(0,-2) node[ground]{};
		\draw (nodo1) to [L]++(2,0) to [short,*-]++(0,0) coordinate (nodo2) to [R]++(0,-2) node[ground]{};
		\draw (nodo2) to [R]++(2,0) to [short,*-]++(0,0) coordinate (nodo3) to [zzDo,invert]++(0,-2) node[ground]{};
		\draw (nodo3) to [short,-o]++(1,0);
	\end{circuitikz}
\end{center}

Ahora agregar componentes a los nodos es sencillo, conectemos un capacitor entre los nodos 1 y 2.

\begin{center}
	\begin{circuitikz}[american voltages]
		\draw (0,0) node[ground]{}; 
		\draw (0,0) to [sV](0,2);
		\draw (0,2) to [R]++(2,0) to [short,*-]++(0,0) coordinate (nodo1) to [C]++(0,-2) node[ground]{};
		\draw (nodo1) to [L]++(2,0) to [short,*-]++(0,0) coordinate (nodo2) to [R]++(0,-2) node[ground]{};
		\draw (nodo2) to [R]++(2,0) to [short,*-]++(0,0) coordinate (nodo3) to [zzDo,invert]++(0,-2) node[ground]{};
		\draw (nodo3) to [short,-o]++(1,0);
		\draw (nodo1) to [short]++(0,1) to [C]++(2,0) to [short]++(0,-1);
	\end{circuitikz}
\end{center}


\subsection{Dibujando Amplificadores Operacionales}
Es importante utilizar coordenadas relativas y etiquetar las coordenadas de los nodos para poder dibujar correctamente los elementos de retroalimentacion.
\begin{center}
	\begin{circuitikz}[american voltages]
		\draw (0,0) node[ground]{}; 
		\draw (0,0) to [V](0,2);
		\draw (0,2) to [R]++(2,0) to [short,*-]++(0,0) coordinate (nodo1) to [R]++(0,-2) node[ground]{};
		\draw (nodo1) to [short]++(1,0) node[op amp, anchor=+] (opamp){};
		\draw (opamp.out) to [R]++(0,-2) node[ground]{};
		\draw (opamp.-) to [short,-*]++(0,1) coordinate (node3);
		\draw (node3) to [R]++(-2,0) coordinate (vo) node[ground]{};
		\draw (node3) to [R](vo -| opamp.out) to [short,-*](opamp.out);		
	\end{circuitikz}
\end{center}




\end{document}